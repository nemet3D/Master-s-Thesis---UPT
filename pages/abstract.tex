\chapter{\abstractname}

%TODO: Abstract
\makeatletter
\ifthenelse{\pdf@strcmp{\languagename}{english}=0}
{\renewcommand{\abstractname}{Rezumat}}
{\renewcommand{\abstractname}{Rezumat în limba engleză}}
Lucrarea prezintă o metodologie de proiectare a paletelor statorului și rotorului pentru o turbină de destindere axială, cu o procedură de optimizare a comportamentului cavitațional.
Tandemul cu palete subțiri stator-rotor este analizat utilizând simularea fluxului incompresibil și inviscid pentru a valida metoda de proiectare cvasi-analitică și programele de calcul asociate. În cele din urmă, grosimea este adăugată la paleta subțire pentru a asigura integritatea structurală și configurația rezultată este analizată în curgere turbulent.

\makeatother

\chapter{\abstractname}

% Undo the name switch
\makeatletter
\ifthenelse{\pdf@strcmp{\languagename}{english}=0}
{\renewcommand{\abstractname}{Rezumat în limba engleză}}
{\renewcommand{\abstractname}{Rezumat}}

The thesis presents a methodology for designing the stator and rotor blades for an axial expansion turbine, with optimisation procedure with respect to cavitation behaviour.
The resulting stator-rotor thin blades tandem is analysed using unsteady, incompressible and inviscid flow simulation in order to validate the quasi-analytical design method and associated computer code. Finally, thickness is added to  the thin blade (camberline) to insure structural integrity and the resulting configuration is analyzed in turbulent flow.

\makeatother
\PassOptionsToPackage{table,svgnames,dvipsnames}{xcolor}

\usepackage[utf8]{inputenc}
\usepackage[T1]{fontenc}
\usepackage{mathptmx}
\usepackage[romanian]{babel} % english is the same as american or USenglish

\usepackage[autostyle]{csquotes}
\usepackage[%
  backend=biber,
  url=true,
  style=numeric, % alphabetic, numeric
  sorting=none, % default == nty, https://tex.stackexchange.com/questions/51434/biblatex-citation-order
  maxnames=4,
  minnames=3,
  maxbibnames=99,
  giveninits,
  uniquename=init]{biblatex} % TODO: adapt citation style
\usepackage{graphicx}
\usepackage{scrhack} % necessary for listings package
\usepackage{listings}
\usepackage{lstautogobble}
\usepackage{tikz}
\usepackage{pgfplots}
\usepackage{comment}
\usepackage{pgfplotstable}
\usepackage{booktabs} % for better looking table creations, but bad with vertical lines by design (package creator despises vertical lines)
\usepackage[final]{microtype}
\usepackage{caption}
\usepackage[unicode]{hyperref}
\usepackage{ifthen} % for comparison of the current language and changing of the thesis layout
\usepackage{pdftexcmds} % string compare to work with all engines
\usepackage{paralist} % for condensed enumerations or lists
\usepackage{subfig} % for having figures side by side
\usepackage{siunitx} % for physical accurate units and other numerical presentations
\usepackage{multirow} % makes it possible to have bigger cells over multiple rows in a table
\usepackage{array} % different options for table cell orientation
\usepackage{makecell} % allows nice manual configuration of cells with linebreaks in \thead and \makecell with alignments
\usepackage{pdfpages} % for including multiple pages of pdfs
\usepackage{adjustbox} % can center content wider than the \textwidth
\usepackage{tablefootnote} % for footnotes in tables as \tablefootnote
\usepackage{threeparttable} % another way to add footnotes as \tablenotes with \item [x] <your footnote> after setting \tnote{x} 

% https://tex.stackexchange.com/questions/42619/x-mark-to-match-checkmark
\usepackage{amssymb}% http://ctan.org/pkg/amssymb
\usepackage{pifont}% http://ctan.org/pkg/pifont
\newcommand{\cmark}{\ding{51}}%
\newcommand{\xmark}{\ding{55}}%


\usepackage[acronym,xindy,toc]{glossaries} % TODO: include "acronym" if glossary and acronym should be separated
\makeglossaries
\loadglsentries{pages/glossary.tex} % important update for glossaries, before document


\bibliography{bibliography}

\setkomafont{disposition}{\normalfont\bfseries} % use serif font for headings
\linespread{1.05} % adjust line spread for mathpazo font

% Add table of contents to PDF bookmarks
\BeforeTOCHead[toc]{{\cleardoublepage\pdfbookmark[0]{\contentsname}{toc}}}

% Settings for pgfplots
\pgfplotsset{compat=newest}

% Settings for lstlistings

% Use this for basic highlighting
\lstset{%
  basicstyle=\ttfamily,
  columns=fullflexible,
  autogobble,
  keywordstyle=\bfseries,
}

% use this for C# highlighting
% %\setmonofont{Consolas} %to be used with XeLaTeX or LuaLaTeX
% \definecolor{bluekeywords}{rgb}{0,0,1}
% \definecolor{greencomments}{rgb}{0,0.5,0}
% \definecolor{redstrings}{rgb}{0.64,0.08,0.08}
% \definecolor{xmlcomments}{rgb}{0.5,0.5,0.5}
% \definecolor{types}{rgb}{0.17,0.57,0.68}

% \lstset{language=[Sharp]C,
% captionpos=b,
% %numbers=left, % numbering
% %numberstyle=\tiny, % small row numbers
% frame=lines, % above and underneath of listings is a line
% showspaces=false,
% showtabs=false,
% breaklines=true,
% showstringspaces=false,
% breakatwhitespace=true,
% escapeinside={(*@}{@*)},
% commentstyle=\color{greencomments},
% morekeywords={partial, var, value, get, set},
% keywordstyle=\color{bluekeywords},
% stringstyle=\color{redstrings},
% basicstyle=\ttfamily\small,
% }

% Settings for search order of pictures
\graphicspath{
    {logos/}
    {figures/}
}

% Set up hyphenation rules for the language package when mistakes happen
\babelhyphenation[english]{
an-oth-er
ex-am-ple
}

% Decide between
%\newcommand{\todo}[1]{\textbf{\textsc{(TODO: #1)}}}} % for one paragraph, otherwise error!
%\newcommand{\done}[1]{\textit{\textsc{(Done: #1)}}}} % for one paragraph, otherwise error!
% and
\newcommand{\todo}[1]{{\bfseries{\scshape{[(TODO: #1)]}}}} % for multiple paragraphs
\newcommand{\done}[1]{{\itshape{\scshape{[(Done: #1)]}}}} % for multiple paragraphs
% for error handling of intended behavior in your latex documents.

\newcommand{\tabitem}{~~\llap{\textbullet}~~}

\newcolumntype{P}[1]{>{\centering\arraybackslash}p{#1}} % for horizontal alignment with limited column width
\newcolumntype{M}[1]{>{\centering\arraybackslash}m{#1}} % for horizontal and vertical alignment with limited column width
\newcolumntype{L}[1]{>{\raggedright\arraybackslash}m{#1}} % for vertical alignment left with limited column width
\newcolumntype{R}[1]{>{\raggedleft\arraybackslash}m{#1}} % for vertical alignment right with limited column width
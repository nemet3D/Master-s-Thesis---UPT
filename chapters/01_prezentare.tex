\section{Utilitate si utilizare}

\subsection{Subsection}

\section{Particularitati constructive}

\subsection{Subsection}

\section{Regimuri de functionare}

\chapter{Prezentare turbine "AXENT"}\label{chapter:prezentare}


\subsection{Istoric al turbinelor axiale de destindere}

\^{I}n ultimii ani, \^{i}n cadrul Institutului de ma\c{s}ini hidraulice din Universitatea Stuttgart, a avut loc cercetarea posibilit\u{a}\c{t}ilor de recuperare a energiei hidraulice \^{i}n anumite situa\c{t}ii speciale din industrie. Astfel a ap\u{a}rut conceptul unei turbine modulare de destindere pentru recuperarea de energie. "Eine modulare axiale Entspannungsturbine" pe scurt AXENT, care este \c{s}i numele comercial al acestei ma\c{s}ini, este rezultatul acestor studii.

Condi\c{t}iile tehnice asumate ini\c{t}ial au presupus g\u{a}sirea de solu\c{t}ii unde avem:

\begin{itemize}
	\item fluid de lucru la presiune ridicat\u{a}
		\begin{itemize}
			\item transport
			\item procese industriale
		\end{itemize}
	\item destinderea presiunii p\^{a}n\u{a} la nivelul dorit
		\begin{itemize}
			\item facut\u{a} prin robine\c{t}i, supape sau arm\u{a}turi
			\item energia poten\c{t}ial\u{a} a presiunii r\u{a}m\^{a}n\^{a}nd nefolosit\u{a}
		\end{itemize}
	\item cre\c{s}terea randamentului prin rec\^{a}\c{s}tigarea de energie
		\begin{itemize}
			\item transformarea energiei hidraulice \^{i}n energie electric\u{a}
			\item far\u{a} a afecta instala\c{t}ia sau procesul deja existent
		\end{itemize}	
\end{itemize}

Pentru situa\c{t}iile enumerate mai sus dorin\c{t}a a fost de a \^{i}ndeplini mai multe lucruri dintre care le voi enumera pe cele mai importante:

\begin{itemize}
	\item proprieta\c{t}i tehnice
		\begin{itemize}
			\item utilizare la debit constant
			\item lipsa \c{s}ocurilor
			\item domeniu de utilizare larg
		\end{itemize}
	\item domenii poten\c{t}iale de utilizare
		\begin{itemize}
			\item alimentare cu ap\u{a}
			\item procese industriale
			\item industria chimic\u{a}
		\end{itemize}
	\item din punct de vedere economic
		\begin{itemize}
			\item recuperare de energie
			\item costuri de realizare \c{s}i intre\c{t}inere c\^{a}t mai mici
		\end{itemize}
	\item obiective politice
		\begin{itemize}
			\item eficientizare ca dorin\c{t}a constant\u{a}
			\item \^{i}mbun\u{a}t\u{a}\c{t}irea bilan\c{t}urilor ecologice
		\end{itemize}	
\end{itemize}

Dup\u{a} analiza tuturor criteriilor pentru care se caut\u{a} o solu\c{t}ie, respect\^{a}nd bine\^{i}n\c{t}eles \c{s}i condi\c{t}iile economice \c{s}i politice, concluzia a fost c\u{a} cel mai bun loc pentru aceast\u{a} viitoare turbin\u{a} este \^{i}n domeniul \textbf{aliment\u{a}rii cu ap\u{a} a ora\c{s}elor mari din zone muntoase.}

\begin{figure}[h!]
	\centering
	\includegraphics[scale=0.8]{figures/amplasare_turbina.jpg}
	\caption{Amplasare turbin\u{a} \cite{andolfatto2016simulation}}
	\label{Amplasare turbin\u{a}}
\end{figure}

Condi\c{t}iile minime care trebuie \^{i}ndeplinite sunt legate de topografia regiunii, pentru a avea o c\u{a}dere \c{s}i o presiune mare, dar \c{s}i m\u{a}rimea re\c{t}elei de alimentare pentru a asigura debitul necesar func\c{t}ion\u{a}rii turbinei la randament ridicat. \^{I}n principiu este necesar\u{a} o topografie deluroas\u{a} \c{s}i alimentarea cu ap\u{a} s\u{a} se fac\u{a} pentru o comunitate de minim 3000 de locuitori.


\subsection{Descrierea turbinelor axiale de destindere}

Turbinele de destindere au ca elemente principale un rotor, c\u{a}ruia datorit\u{a} mi\c{s}c\u{a}rii apei \^{i}i este imprimat\u{a} o mi\c{s}care de rota\c{t}ie \c{s}i un generator pentru transformarea lucrului mecanic \^{i}n energie electric\u{a}. Un punct dup\u{a} care putem categorisi turbinele axiale de destindere este pozi\c{t}ia generatorului, varianta "Inline" (vezi Figura 6) cu generator imersat \^{i}n conduct\u{a} imediat dup\u{a} rotor sau cu generator aflat \^{i}n afara conductei de trecere a apei, generator exterior (Figura 7).

\begin{figure}[h!]
	\centering
	\includegraphics[scale=1.25]{figures/generator_inline.jpg}
	\caption{Generator imersat "Inline" \cite{GREES_2014}}
	\label{Generator imersat Inline}
\end{figure}

\begin{figure}[h!]
	\centering
	\includegraphics[scale=1.25]{figures/generator_in_afara_conductei.jpg}
	\caption{Generator \^{i}n afara conductei \cite{GREES_2014}}
	\label{Generator \^{i}n afara conductei}
\end{figure}

Dup\u{a} cum se vede \c{s}i \^{i}n ultimele imagini, avantajul turbinei cu generator "Inline" const\u{a} \^{i}n capacitatea de instalare modular\u{a} mult mai simpl\u{a}, prin sec\c{t}ionarea conductei \c{s}i prinderea modulului turbinei prin flan\c{s}e \^{i}n conduct\u{a}.

\begin{figure}[h!]
	\centering
	\includegraphics[scale=0.6]{figures/modul_singular.png}
	\caption{Modul singular \cite{susanhub}}
	\label{Modul singular}
\end{figure}

\^{I}n proiectarea modern\u{a} a turbinelor axiale, se folose\c{s}te frecvent teoria hidraodinamic\u{a} a re\c{t}elelor de profile. Metodele numerice de determinare a re\c{t}elelor de profile, care exprim\u{a} legatur\u{a} \^{i}ntre parametrii geometrici \c{s}i cei hidrodinamici au fost folosite \^{i}n cadrul Universit\u{a}\c{t}ii Suttgart pentru a ajunge la un produs cu aplicabilitate imediat\u{a} \c{s}i tangibil\u{a}. Rezultatele acestor cercet\u{a}ri au fost dezvoltarea unei turbine cu aparat director format din 26 profile \c{s}i un rotor cu 30 palete, care se poate folosi \^{i}n modul singular sau \^{i}n dou\u{a} trepte (vezi Figura 10 \c{s}i Figura 11).

\begin{figure}[h!]
	\centering
	\includegraphics[scale=0.6]{figures/modul_in_doua_trepte.png}
	\caption{Modul \^{i}n dou\u{a} trepte \cite{susanhub}}
	\label{Modul \^{i}n dou\u{a} trepte}
\end{figure}



\subsection{Alte solu\c{t}ii. Turbin\u{a} axial\u{a} cu rotoare contrarotative}

Conceptul de micro turbin\u{a} contrarotativ\u{a} prezentat \^{i}n Figura 10 \c{s}i descris \^{i}n detaliu \^{i}n lucr\u{a}rile \cite{andolfatto2016simulation} \c{s}i \cite{andolfatto2015mixed} este un candidat pentru recuperarea energiei din re\c{t}elele cu ap\u{a} potabil\u{a}, chiar \c{s}i \^{i}n locurile cu putere disponibil\u{a} \^{i}ntre 5 kW \c{s}i 25 kW. Prezint\u{a} o arhitectur\u{a} axial\u{a} compact\u{a} asigur\^{a}nd posibilitatea instal\u{a}rii in-line \^{i}n re\c{t}elele deja existente, limit\^{a}nd astfel cheltuielile financiare \c{s}i impactul asupra mediului \c{s}i infrastructurii.

Aceste ma\c{s}ini opereaz\u{a} la viteze variabile \c{s}i acoper\u{a} o arie ridicat\u{a} de operabilitate. Folosind o vitez\u{a} variabil\u{a} \^{i}ntre cele dou\u{a} rotoare ridic\u{a} de asemenea eficien\c{t}a total\u{a} a turbinei, ridic\^{a}nd astfel veniturile a\c{s}teptate.

\begin{figure}[h!]
	\centering
	\includegraphics[scale=0.75]{counter_rotating_runners}
	\caption{Turbina contrarotativ\u{a} \cite{andolfatto2016simulation}}
	\label{Turbina contrarotativ\u{a}}
\end{figure}

\clearpage


\subsection{Utilizarea modular\u{a}}
Unul din marile avantaje al acestor turbine axiale de destindere este posibilitatea de a le folosi \^{i}n serie, forma lor compact\u{a} \c{s}i cotele de gabarit reduse \^{i}ncuraj\^{a}nd acest lucru.

\begin{figure}[h!]
	\centering
	\includegraphics[scale=0.9]{modular}
	\caption{Modular \cite{hasmatuchi2014new}}
	\label{Modular}
\end{figure}

\subsection{Descrierea elementelor componente}

Principalele elemente componente ale turbinei axiale de destindere sunt:

\begin{itemize}
	\item \textbf{statorul}; conduce apa spre rotor \c{s}i asigur\u{a} vitezele respectiv circula\c{t}ia necesar\u{a} transform\u{a}rii energetice optime, \^{i}n condi\c{t}iile pierderilor hidraulice minime.
	\item \textbf{rotorul}; are rolul de a transforma energia hidraulic\u{a} disponibil\u{a} \^{i}n energie cinetic\u{a}, conduc\^{a}nd apa care vine prin stator \^{i}n tubul de aspira\c{t}ie al turbinei.
	\item \textbf{generatorul electric}; este un dispozitiv care transform\u{a} energia mecanic\u{a} \^{i}n energie electric\u{a}.
\end{itemize}


\subsection{Parametrii fundamentali ai turbinelor}

Turbinele hidraulice se pot caracteriza din punct de vedere func\c{t}ional prin folosirea urm\u{a}torilor parametri:

\begin{itemize}
	\item debitul \textit{Q, $[\si{m}]$}
	\item c\u{a}derea \textit{H, $[\si{m}]$}
	\item puterea \textit{$P_m$, $[\si{W}]$}
	\item viteza unghiular\u{a} \textit{\(\omega\), $[\si{rad/s}]$}
	\item randamentul \textit{\(\eta\), $[-]$}
	\item \^{i}n\u{a}l\c{t}imea geometric\u{a} de aspira\c{t}ie \textit{\(h_s\), $[\si{m}]$}
	\item coeficientul de cavita\c{t}ie \textit{\(\sigma\), $[-]$}
\end{itemize}


\subsubsection{Debitul}

Definim debitul volumic ca volumul de ap\u{a} ce intr\u{a} \^{i}n turbin\u{a} \^{i}n unitatea de timp. Se exprim\u{a} \^{i}n unit\u{a}\c{t}i de volum, greutate sau de mas\u{a}, raportate la unitatea de timp.


\subsubsection{C\u{a}derea turbinei}

Consider\^{a}nd sec\c{t}iunea de la intrarea \^{i}n turbin\u{a} ca notat\u{a} cu i \c{s}i ie\c{s}irea cu e, c\u{a}derea turbinei se exprim\u{a} sub forma $H=e_i-e_e$, deci

\begin{equation}
H=\bigg(\frac{p_i}{{\rho}g}+z_i+\frac{v_i^2}{2g}\bigg)_i-\bigg(\frac{p_e}{{\rho}g}+z_e+\frac{v_e^2}{2g}\bigg)_e \;[\si{m}]
\end{equation}

unde: $p\;[\si{Pa}]$ este presiunea, ${\rho}\;[\si{kg/m^3}]$ este densitatea apei, $z\;[\si{m}]$ este \^{i}n\u{a}l\c{t}imea, $v\;[\si{m/s}]$ este viteza iar $g\;[\si{m/s^2}]$ este accelera\c{t}ia gravita\c{t}ional\u{a}.

\subsubsection{Puterea}

Puterea hidraulic\u{a} sau puterea sursei $P_h$ este puterea disponibil\u{a} a apei de la intrarea \^{i}n turbin\u{a} pentru a putea fi transformat\u{a} \^{i}n putere mecanic\u{a} la arborele turbinei:

\begin{equation}
P_h={\rho}gQH\; [\si{W}]
\end{equation}

Puterea mecanic\u{a} a turbinei $P_m$ este puterea mecanic\u{a} la arborele turbinei, ob\c{t}inut\u{a} din momentul la arbore $M$ \c{s}i viteza unghiular\u{a} $\omega$:

\begin{equation}
P_m=M{\omega}\; [\si{W}]
\end{equation}

\subsubsection{Tura\c{t}ia}

Rota\c{t}ia sau tura\c{t}ia turbinei este num\u{a}rul de \^{i}nv\^{a}rtituri realizate \^{i}n unitatea de timp, de obicei \^{i}ntr-un minut (uneori \^{i}n secund\u{a}). Se noteaz\u{a} cu \textit{n} [rpm] sau cu \(\omega\) rad/s (SI).

\begin{equation}
n\; [\si{rpm}]\rightarrow{\omega}\; [\si{rad/s}]=\frac{{\pi}n}{30}
\end{equation}


\subsubsection{Randamentul}

\^{I}n turbina hidraulic\u{a} transmiterea energiei de la curentul de ap\u{a} c\u{a}tre rotor se efectueaz\u{a} cu pierderi. Aceastea sunt definite \c{s}i m\u{a}surate prin randamentul turbinei:

\begin{equation}
{\eta}=\frac{Putere\; utila}{Putere\; consumata}=\frac{P_m}{P_h}=\frac{M{\omega}}{{\rho}gQH}
\end{equation}

\begin{comment}
\subsubsection{In\u{a}ltimea geometric\u{a} de aspiratie}

\^{I}n\u{a}l\c{t}imea geometric\u{a} de aspira\c{t}ie este diferen\c{t}a \^{i}n plan vertical dintre o referin\c{t}\u{a} a turbinei (de obicei planul care trece prin axa paletelor rotorului) \c{s}i planul nivelului apei din aval.

\subsubsection{Coeficientul de cavita\c{t}ie}

Cavita\c{t}ia este un fenomen hidrodinamic complex, caracterizat de apari\c{t}ia, dezvoltarea \c{s}i surparea rapid\u{a} a unor bule cavita\c{t}ionale, umplute cu aburi \c{s}i gaze, ce apare \^{i}ntr-un curent de lichid \^{i}n zona cu viteze mari \c{s}i presiuni mici. Procesul fizic numit cavita\c{t}ie se apropie mai mult sau mai pu\c{t}in de acela al fierberii lichidelor. Efectele principale ale cavita\c{t}iei sunt: vibra\c{t}ii \c{s}i zgomote puternice, distrugerea materialului componentelor turbinei \c{s}i scaderea randamentului de func\c{t}ionare.

Aceste fenomene se intensific\u{a} atunci cand pragul de cavita\c{t}ie limita este depasit, sau se func\c{t}ioneaz\u{a} \^{i}n supercavita\c{t}ie. Profesorul D.Thoma a introdus pentru prima oara no\c{t}iunea de prag de cavita\c{t}ie. Acest criteriu este denumit coeficient de cavita\c{t}ie \c{s}i se exprim\u{a} \^{i}n felul urmator:

\begin{equation}
\sigma=\frac{A-A\pm{H_s}}{H}
\end{equation}

\end{comment}


\subsection{Valori tipice pentru parametrii fundamentali ai turbinelor de destindere}

\^{I}n literatura de specialitate sunt investigate turbinele de destindere cu valori fundamentale care se \^{i}ncadreaz\u{a} deseori \^{i}ntre anumite limite.


\subsubsection{Debitul}
Pentru debitul volumetric valorile generale pe care le g\u{a}sim \^{i}n articolele stiin\c{t}ifice sunt urm\u{a}toarele:

\begin{itemize}
	\item 0.2500$\si{m^3/s}$ \cite{gentner2000experimentelle}
	\item 0.1200-0.1900$\si{m^3/s}$ \cite{GREES_2014}
	\item 0.0500-0.1450$\si{m^3/s}$ \cite{susanhub}
	\item 0.0090-0.0140$\si{m^3/s}$ \cite{biner2016engineering}
	\item 0.0375$\si{m^3/s}$ \cite{hasmatuchi2014new}
\end{itemize}


\subsubsection{C\u{a}derea}

\begin{itemize}
	\item 50$\si{m}$ \cite{gentner2000experimentelle}
	\item 40-200$\si{m}$ \cite{GREES_2014}
	\item 10.2-205.4$\si{m}$ \cite{susanhub}
	\item 24.265-61.162$\si{m}$ \cite{biner2016engineering}
	\item 29.9$\si{m}$ \cite{hasmatuchi2014new}
\end{itemize}


\subsubsection{Tura\c{t}ia}

\begin{itemize}
	\item 1500$\si{rpm}$ \cite{gentner2000experimentelle}
	\item 3000$\si{rpm}$ \cite{GREES_2014}
	\item 1500 si 3000$\si{rpm}$ \cite{susanhub}
	\item 3000$\si{rpm}$ \cite{biner2016engineering}
	\item 3000$\si{rpm}$ \cite{hasmatuchi2014new}
\end{itemize}

\^{I}n continuare, valorile alese pentru desf\u{a}\c{s}urarea analizei \c{s}i proiect\u{a}rii turbinei axiale de destindere vor fi cele rezultate experimental \^{i}n lucrarea \cite{susanhub} \c{s}i anume: \textbf{debitul volumetric de 0.070 si 0.135$\si{m^3/s}$, c\u{a}derea de 24.0 respectiv 87.6$\si{m}$ \c{s}i tura\c{t}ia de 1500 \c{s}i 3000$\si{rpm}$ la un diametru $D_p$ de 220$\si{m}$ la periferie \c{s}i $D_b$ 190$\si{m}$ la butuc.}.

\clearpage
\chapter{Determinarea diametrului la butuc}\label{chapter:butuc}

\section{Calculul diametrului la butuc}

Pentru a calcula diametrul la butuc conform teoriei dezvoltata la Timișoara de Susan-Resiga et al \cite{susanhub} trebuie sa rezolvam ecuația pentru intensitatea rotației $\sigma$, cu un rezultat adimensional.

Conform ecuației fundamentale a turbomașinilor (ecuația lui Euler) avem viteza tangențială la periferie:

\begin{equation}
gH=UV_{u}, \text{ sau } gH=\frac{\pi n}{30} R_{p} V_{up} \Rightarrow V_{u}=\frac{30gH}{\pi n R_{p}}
\end{equation}

Viteza axiala medie (debitanta) prind conductă este:

\begin{equation}
V_{ad}=\frac{Q}{\pi R_{p}^2}
\end{equation}

În ambele ecuații, (2.1) și (2.2) raza conductei este practic raza de la periferie a turbinei. Ca rezultat, parametrul de intensitate a rotației este:

\begin{equation}
\sigma \equiv \frac{V_{up}}{V_{ad}} = 30 \frac{g H R_{b}}{n Q} = 7.4
\end{equation}

În care regiunea (necunoscută) stagnantă $R_{s}$, se indică raportul de raza de stagnare fata de raza la periferie, totul la pătrat, de unde putem obține $x$ și apoi raza la butuc:

\begin{equation}
\text{Definim } x \equiv \bigg(\frac{R_{s}}{R_{p}}\bigg)^2 \Rightarrow R_{s} = \sqrt{x} R_{p}
\end{equation}

Pentru o turbionare cu circulație constantă și o presiune totală constantă în conducta a fost derivata \cite{susanhub} o ecuație algebrică pentru $x$, dedusa cu parametrul $\sigma$:

\begin{equation}
\frac{\sigma^2}{x^2} - \frac{2}{(1-x^3)} = 0
\end{equation}

Din (2.4) și (2.5) putem calcula raza de stagnare $R_{s}$ folosind metoda dezvoltata de Lebedev \cite{lebedev1991formulae} și obținem o valoare de $94\si{mm}$.

\clearpage
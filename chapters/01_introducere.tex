\chapter{Introducere}\label{chapter:introducere}

În opinia multor experți, creșterea eficientei energetice este, în viitor, una din căile de abordare spre o reducere sau măcar o limitare a cerinței globale de energie. Multe procese industriale au nevoie de fluide la presiune înaltă. La sfârșitul proceselor, fluidele sunt adesea degajate la presiunea mediului înconjurător prin robineți convenționali. În aceste cazuri, energia potențială a fluidului este disipată și rămâne nefolosită.

O turbina axială de destindere este o posibilitate de recuperare, de convertire a presiunii în energie electrică. Refolosirea energiei recuperate creste eficiența totală a sistemelor.

\textbf{Scopul} acestei lucrări de disertație este de a prezenta o metodologie de proiectare a paletelor statorului și rotorului specifică pentru o turbină axiala de destindere, cu o procedură de optimizare a comportamentului cavitațional.
Tandemul cu palete subțiri stator-rotor este analizat utilizând simularea fluxului incompresibil și inviscid pentru a valida metoda de proiectare cvasi-analitică și programele de calcul asociate. În cele din urmă, grosimea este adăugată la paleta subțire pentru a asigura integritatea structurală și configurația rezultată este analizată în curgere turbulent.